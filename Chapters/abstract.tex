\addtotoc{Abstract} % Add the "Abstract" page entry to the Contents
\abstract{\addtocontents{toc}{\vspace{1em}} 

The need to secure communications between personal devices is increasing nowadays, especially in the context of Internet of Things. Internet of Thing (IoT) extensively uses wireless communication so that all devices can work together without being physically attached. However, securing these communications requires dedicated protocols in particular in term of device authentication and neighbour discovery. Such protocols use classical cryptographic mechanisms but also rely on assumptions about physical characteristics (existence of an out-of-band or human assisted channel, location or signal range of devices, ...). Additionally, they are not sufficiently effective on computation and communication on constrained devices in IoT. In mean while, formally proving security properties for these protocols is another challenge: most of existing security protocol verification approaches do not or partially take into account these physical characteristics. Furthermore, the attacker model based on Dolev-Yao used in these approaches is no longer suitable.

This thesis propose a new device pairing protocol proved in our model that is more efficient than other competitors in term of communication cost. To tackle above problem on formal verification, this thesis conducts a new formal model that extends Strand Space model with physical characteristics and a refined penetrator model, and to apply it to analyse device pairing and neighbourhood discovery protocols. Thanks to this approach, we identify new security flaws in several protocols, that has not be published before. We then propose a translation procedure that transforms a model in our formalism of an initial protocol with out-of-band channels into a model in original Strand Spaces of a protocol that does not use any OOB channel. This translation preserves security properties of the initial protocol so that it can be automatically checked using existing security protocol verification tools. Based on above mentioned enhanced security blocks, we define a new promising secure and robust bootstrapping scheme as our last contribution. Our scheme enables a new resource constrained device to securely join a home network in circumstances where the home-gateway is down, or the device is second-hand. Furthermore, our scheme does not require pre-shared symmetric or public keys implanted in the device at the manufacture site, nor does it require a PKI infrastructure.

}

\vspace{.6cm}
{\bf KEY-WORDS: } Formal Model, Security Verification, Authentication Protocols, Strand Space, Internet of Things, Out-of-Band Channels
