% Chapter Template

\chapter{Extension of Strand Spaces for Wireless Protocols} % Main chapter title

\label{Chapter2} % Change X to a consecutive number; for referencing this chapter elsewhere, use \ref{ChapterX}

\lhead{Chapter 2. \emph{Extensions of Strand Spaces for Wireless Protocols}} % Change X to a consecutive number; this is for the header on each page - perhaps a shortened title

%----------------------------------------------------------------------------------------
%	SECTION 2
%----------------------------------------------------------------------------------------

 
This chapter presents background on original Strand Spaces helpful for understanding our extensions. Introduced in the previous chapter, Strand Spaces model is chosen as a strong floor for us to build our model up, thanks to its advantage. Thus, we explain the fundamental theory and some its notable extensions in detail. The core theory then is broadened with our extensions on algebra, physical properties, and a new penetrator model. In particular, a new strand, so called a wireless strand, is conducted by notations of location, timestamp, signal range, and channel. 

The chapter begins with the original Strand Spaces theory proposed by Gutman et all in 1997. Some extension for Strand Spaces are also included since we will use them in our work. Finally, our extension, so called wireless Strand Spaces, is given, and our extended penetrator model is presented as well. 

\section{Strand Spaces Model}

In 1997, Fabrega, Herzog and Gutman developed a new method to prove a protocol if it achieves authentication and secrecy properties or not. First published internally as a technical report, the work went out public in a conference paper ~\cite{674832} in a year later. Following this approach, the authors continued maturing their model. Basic Strand Spaces theory was extended with \emph{honest idea}~\cite{Thayer:1998:HIS:794198.795096} which allows to learn general principles that limit penetrators' capabilities. After that, \emph{mixed strands}~\cite{Thayer:1999:MSS:794199.795113} was proposed to study problems of mixed protocols. One huge milestone of Strand Spaces theory is \emph{authentication tests}~\cite{Guttman:2002:ATS:568264.568267} proposed in 2000. In a study of cryptographic protocols, authors realized that after emitting a message containing a new value, a participant receives a cryptographic form of this value, this must exist a regular participant of the protocol sent it. This scheme, so called an authentication test, works as a powerful tool to minimise work of proving, and straightforwardly give results of authentication protocols. From 2002 to 2007, Strand Spaces theory was enriched with \emph{shape of bundle, skeleton and homomorphism}~\cite{Doghmi:2007:SHS:1230146.1230260}. These theories study about a sort of protocols that share the same forms. Current developments of Strand Spaces are focusing on various of types of protocols such as TLS~\cite{Kamil:2011:ATS:2590701.2590707}, location-awareness protocols~\cite{Thayer:2010aa}, and DH protocols~\cite{1212716}. 

In our work, we continue adding more ingredients in Strand Spaces to get more juice in wireless context. We select and present some extensions that we will use in our model. Other extensions objective closely to our work will be presented in related work of chapter 3 and 4. Definitions used in thesis are referred from \cite{674832}, \cite{Guttman:2002:ATS:568264.568267} and~\cite{Doghmi:2007:SHS:1230146.1230260}. 

\subsection{Fundamental Theory}

A security protocol is an ordered sequence of messages that participants exchange in a protocol. Let $\mathcal{A}$ a set of possible messages intentionally transferred in a security protocol, and elements of this set are referred to as \textit{terms}. Additionally, the set $\mathcal{A}$ is algebra freely generated from of a set of text terms $\mathcal{T}$ and a set of cryptographic keys $\mathcal{K}$ by means of concatenation and encryption. While $\mathcal{T}$ contains textual information such as nonces, $\mathcal{K}$ contains symmetric and/or asymmetric cryptographic keys. The two sets $\mathcal{T}$ and $\mathcal{K}$ are disjoint. 

\begin{Definition} $\emph{Compound terms}$ are generated by two operators
\begin{itemize}
	\item encr: $\mathcal{K} \times \mathcal{A} \rightarrow \mathcal{A}$ representing encryption
	\item join: $\mathcal{A} \times \mathcal{A} \rightarrow \mathcal{A}$ representing concatenation
\end{itemize}
\end{Definition}

For convention, from now we express the concatenation of two term $t$ and $t'$ as $\textit{tt'}$ and encryption of term $t$ with key $K$ as $\{t\}_K$. We continue defining a relation, \textit{subterm relation}, on the set $\mathcal{A}$. 

\begin{Definition} The $\emph{subterm relation}$ $\sqsubseteq$ over $\mathcal{A}$ is defined inductively as: 
	\begin{itemize}
		\item $a \sqsubseteq t $ for $t \in \mathcal{T}$   iff $a = t $
		\item $a \sqsubseteq key $ for $t \in \mathcal{K}$   iff $a = key $
		\item $a \sqsubseteq gh $ iff $a \sqsubseteq gh, a \sqsubseteq h $ or $ a= h $
		\item $a \sqsubseteq \{g\}_K $ iff $a \sqsubseteq g $ or $a = \{g\}_K$
	\end{itemize} 
\end{Definition}

Remark that, $t_1\sqsubseteq t$ means $t_1$ is a subterm of $t$. A \textit{subterm} is just a term that can be easily extracted from a term with an appropriate key. Taking the Needham-Schroeder \cite{Needham:1978:UEA:359657.359659}(NS) protocol as an example, initiator A starts sending a message (or a $term$) of the form $\{N_a,A\}_{PK_B}$ to intentional responder B, where $PK_B$ is the public key of B. Subterms of term $\{N_a,A\}_{PK_B}$ could be $N_a, A$, or $\{N_a,A\}_{PK_B}$. Since $PK_B$ is a key, $PK_B \not\sqsubseteq \{N_a,A\}_{PK_B}$, except a case when $PK_B$ is in a value of this message. 

Terms are extended to $\textit{signed term}$ including a positive represented to a transmission, and a negative one represented to a reception. 

\begin{Definition} A $\emph{signed term}$ is a pair $\langle\delta,a\rangle$ with $a \in A$ and $\delta$ one of the symbols $+,-$. We will write a signed term as $+t$ or $-t$. $(\pm A)^*$ is the set of finite sequences of signed terms. We denote a typical element of $(\pm A)^*$ by $\langle \delta_1,a_1\rangle,.....,\langle\delta_n,a_n\rangle$.
\end{Definition}

A sequence of sending and receiving messages in a protocol is called $strand$, and a set of strands is called $\textit{Strand Spaces}$. 

When modeling a protocol, the strand of a principal is a sequence of events as seen by that principal in a particular protocol run and in a particular instance of role. For example, the strand of the initiator for the NS protocol is $\langle+\{N_a,A\}_{PK_B}, -\{N_a,N_b\}_{PK_A},+\{N_b\}_{PK_B} \rangle$. The first event is emission of $\{N_a,A\}_{PK_B}$ followed by reception of $\{N_a,N_b\}_{PK_A}$ and so on. 

\begin{Definition}
A $\emph{Strand Spaces}$ is a set $\Sigma$ with a trace mapping tr: $\Sigma \rightarrow (\pm A)^*$.
\end{Definition}

For a Strand Spaces $\Sigma$:

	\begin{itemize}
		\item A node is a pair $\langle s,i\rangle$, with $s \in \Sigma$ and i an integer satisfying $1\leq i \leq length(tr(s))$. The set of nodes is denoted by $\mathcal{N}$. We will say the node $n= \langle s,i\rangle$ belongs to the strand s. Clearly, every node belongs to a unique strand. 

		\item If $n= \langle s,i \rangle \in \mathcal{N}$ then $index(n) = i$ and $strand(n)=s$. Define $term(n)$ to be $(tr(s))_i$, i.e. the $i^{th}$ signed term in the trace of $s$. Similarly, $uns\_term(n)$ is $((tr(s))_i)_2$, i.e. the unsigned part of the $i^{th}$ signed term in the trace of $s$

		\item If $n,n' \in \mathcal{N}, n \rightarrow n'$ mean $term(n) = +a$ and $term(n') = +a$ It means that node $n$ sends the message $a$ which is received by $n'$, creating a causal link between their strands.

		\item If $n,n' \in \mathcal{N}, n \Rightarrow n'$ mean $n$ and $n'$ occur on the same strand with $index(n)=index(n') -1 $. It expresses that $n$ is an immediate causal predecessor of $n'$ on the strand. 
		
		\item $n \Rightarrow^+ n'$ to mean that $n$ precedes $n'$ (not necessarily immediately) on the same strand.
			
		\item $m \Rightarrow^* n$ means $m \rightarrow n_1 \Rightarrow m_2 \rightarrow n_2 \Rightarrow ......m_k \rightarrow n$ where $n_i, m_i$, and n stand on the same or different strands. A \emph{path} created by $m \Rightarrow^* n$ forms a connectivity subgraph in the bundle. 

		\item An unsigned term $t$ $occurs$ in $n \in \mathcal{N}$ if $t \sqsubseteq term(n)$.

		\item An unsigned term $t$ $originates$ on $n \in \mathcal{N}$ iff:$term(n)$ is positive, $t \sqsubseteq term(n)$, and whenever $n$ precedes $n$ on the same strand, $t \not\sqsubseteq term(n)$.

		\item An unsigned term $t$ is $\textit{uniquely originating}$ iff $t$ originates on a unique $n \in \mathcal{N}$ 

	\end{itemize}
	
Let $\mathcal{N}$ be a set of nodes, and let $\mathcal{E}$ be the union of the sets of $\rightarrow$ and $\Rightarrow$ edges. A directed graph $\mathcal{G}$ has a structure $\mathcal{G} = \langle \mathcal{N},\mathcal{E}\rangle$. A $bundle$ is a finite subgraph of this graph in which the edges express casual dependencies of the nodes.

\begin{Definition} Suppose $\mathcal{N_B}$ be a subset of $\mathcal{N}$, and $\mathcal{E_B}$ be a subset of $\mathcal{E}$. Let $\mathcal{B} = \langle \mathcal{N_B},\mathcal{E_B}\rangle$ be a subgraph of $\mathcal{G}$. $\mathcal{B}$ is a $\emph{bundle}$ if :
\begin{enumerate}
\item $\mathcal{N_B}$ and $\mathcal{E_B}$ are finite
\item If $n \in \mathcal{N_B}$ and $term(n)$ is negative, then there is a unique $n'$ such that $ n' \rightarrow n \in \mathcal{E_B}$
\item If $n \in \mathcal{N_B}$ and $n \Rightarrow n'$ then $n \Rightarrow n' \in \mathcal{E_B}$ 
\item $\mathcal{B}$ is acyclic
\end{enumerate}
\end{Definition}

Remark: Guttman implicitly expressed that when a node transmits a message, there is more than one (or none) receiving node of the message. However, we hardly see more than two receptions in their studies. 

\begin{Definition}
A node $n$ belongs to a bundle $\mathcal{B} = \langle \mathcal{N_B},\mathcal{E_B}\rangle$ , written $n \in \mathcal{B}$ if $n \in \mathcal{N_B}$. The $\mathcal{B} -height$ of a strand $s \in \mathcal{B}$ is the largest $i$ such that $\langle s,i \rangle \in \mathcal{B}$. 
\end{Definition}

For instance, in NS protocol, $\mathcal{B}-height$ of initiator strand is 3, similarly $\mathcal{B}-height$ of responder strand is also 3.
%introduce component
\subsection{Component, Authentication Tests}

After introducing Strand Spaces fundamental theory, basing on the fact that security authentication protocols usually use specific challenge-response methods to obtain goals, Thayer et al\cite{authenticationtests} continued publishing a theory of $\textit{authentication tests}$ as a tool to prove security properties for protocols easily. 

\begin{Definition}
 A term $t'$ is called a \emph{component} of term $t$ if $t'$ cannot be split to another term $t''$, and $t$ is built by concatenating $t'$ with arbitrary terms.
\end{Definition}

Conveniently, we refer the example of Thayer\cite{Thayer:2010aa}. If we have a term like $B\{N_aK\{KN_b\}_{K_B}\}_{K_A}N_a$, then it contains three components: $B, \{N_aK\{KN_b\}_{K_B}\}_{K_A}$, and $N_a$. 
 
\begin{Definition}
 For a strand s, a term t is \emph{new} at $n = \langle s,i\rangle$ if t is a component of term(n), but t is not a component of node $\langle s,j\rangle$ for every $j<i$. 
\end{Definition}

\begin{Definition} Suppose that $n \in \mathcal{B}$ is positive, $\emph{a}$ $\in \mathcal{A}$ is a subterm of $term(n)$. The edge $n \Rightarrow^+ n'$ is a \emph{transformed edge} for $\emph{a}$ if there exits a negative node $n' \in \mathcal{B}$, and there is a new component $t_2$ of $n'$ such that $a \sqsubseteq t_2$.
\end{Definition}

Respectively, \textit{a transforming edge} is denoted. 

\begin{Definition} Suppose that $n \in \mathcal{B}$ is negative, $\emph{a}$ $\in \mathcal{A}$ is a subterm of $term(n)$. The edge $n \Rightarrow^+ n'$ is a \emph{transforming edge} for $\emph{a}$ if there exits a positive node $n' \in \mathcal{B}$, and there is a new component $t_2$ of $n'$ such that $a \sqsubseteq t_2$.
\end{Definition}

\begin{Definition} 
The edge $n \Rightarrow^+ n'$ is \emph{a test} for $\emph{a}$ $\in \mathcal{A}$ if $\emph{a}$ uniquely originates at $n$, and $n \Rightarrow^+ n'$ is a transformed edge for $\emph{a}$. 
\end{Definition}

\begin{Definition} Suppose that $n, n' \in \mathcal{B}$.
\begin{enumerate}
\item The edge $n \Rightarrow^+ n'$ is \emph{a outgoing test} for $a$ $\sqsubseteq t = \{|h|\}_K$ if it is a test for $a$ in which $K^{-1} \not\in P$, $a$ does not occur in any component of $n$ other than t. Moreover, $t$ is a test component for $a$ in $n$.
\item The edge $n \Rightarrow^+ n'$ is \emph{a incoming test} for $a$ $\sqsubseteq t_1 = \{|h|\}_K$ if it is a test for $a$ in which $K \not\in P$, and $t_1$ is a test component for $a$ in $n'$.
\end{enumerate}
\end{Definition}

Subsequently, authentication tests~\cite{authenticationtests} are provided as powerful and simple tools to guarantee existence of regular strands in a bundle. 

\emph{Authentication Test 1}: Suppose that $n' \in \mathcal{B}$, and $n \Rightarrow^+ n'$ is outgoing test for $\emph{a}$ $\sqsubseteq t$ with $t = term(n)$. Then there exist regular nodes $m,m' \in \mathcal{B}$ such that $t$ is a component of $m$, and  $m \Rightarrow^+m'$ is a transforming edge for $\emph{a}$. In addition that $\emph{a}$ occurs only in component $t_1=\{|h_1|\}_{K_1}$ of $m'$, that $t_1$ is not a proper subterm of any regular component, and that $K^{-1}_1 \not\in P$. There is a negative regular node with $t_1$ as a component. 


\emph{Authentication Test 2}: Suppose that $n \in \mathcal{B}$, and $n \Rightarrow^+ n'$ is incoming test for $\emph{a}$ $\sqsubseteq t'$ with $t' = term(n')$. Then there exist regular nodes $m,m' \in \mathcal{B}$ such that $t'$ is a component of $m'$, and  $m \Rightarrow^+m'$ is a transforming edge for $\emph{a}$. 

\begin{Definition} 
A negative node is an \emph{unsolicited test} for $t = \{|h|\}_K$ if $t$ is a test component for any $a$ in $n$ and $K \not\in P$. 
\end{Definition}

\emph{Authentication Test 3}: Suppose that a node $n$ is in a bundle $\mathcal{B}$, and $n$ be an unsolicited test for $t = \{|h|\}_K$, then there exists a positive regular node $m \in \mathcal{B}$ such that $t$ is a component of $m$. 
 
The proofs of these authentication tests are out of scope in this part. So if readers eager to deeply understand the proofs, please regard to the paper \cite{Guttman:2002:ATS:568264.568267}.


\subsection{Shape and Skeleton}

In design protocol, ones always desire that there is only one possible execution of their protocol in any scenario. Nevertheless, there might exist some executions relative to the protocol's assumptions. Actually, the executions of protocols normally have very few essentially different forms that called $shapes$. Then, authentication and secrecy properties can be determined by examining the shapes.

Precisely, a shape is a local execution by honest principals. Partial information about a principal's execution of a protocol is called \emph{skeleton}. Skeletons are partial-ordered structures, or fragments of message sequence chart. Moreover, a skeleton is $realised$ if it is not fragmented, i.e. it contains exactly the regular behavior of some executions. A realised skeleton is a shape if it is minimal. 

A preskeleton describes the regular parts of a set of bundles. Formally presented, preskeleton is defined as follows.
\begin{Definition}[Skeleton]A four-tuple $\mathcal{A} = (node, \preceq, non, unique)$ is a preskeleton if:
	\begin{enumerate}
	\item $node$ is a finite set of regular nodes, $n_1 \in node$ and $n_0 \Rightarrow^+ n_1$ implies $n_0 \in nodes$;
	\item $\preceq$ is a partial ordering on $node$ such that $n_0 \Rightarrow^+ n_1$ imples $n_0 \preceq n_1$;
	\item $non$ is a set of keys where if $K \in non$, then for all $n \in node, K\not\sqsubseteq term(n)$, and for some $n' \in node$, either $K$ or $K^{-1}$ is used in $term(n')$;
	\item $unique$ is a set of atoms where $a \in unique$, for some $n \in node, a \sqsubseteq term(n)$. 
	\end{enumerate}
	A preskeleton $\mathcal{A}$ is a $skeleton$ if in addition:
	\begin{enumerate}
	\item[4'.] $a\in unique$ implies $a$ originates at no more than one $n\in node$. 
	\end{enumerate}
\end{Definition}

\begin{Definition}[Shape] $\mathcal{A}'$ is a $shape$ for $\mathcal{A}$ if (1) some $H : \mathcal{A} \rightarrow \mathcal{A'}$, (2) $\mathcal{A'}$ is realised, and (3) no proper skeleton of $\mathcal(A')$ satisfies (1) and (2). 
\end{Definition}

\subsection{Penetrator Model}

Original Strand Spaces theory uses Dolev-Yao \cite{dolev-yao} model as its penetrator model. Penetrator's power is built up from two ingredients: initially known keys available to the penetrator, and actions that allows the penetrator manipulates messages. The actions are summarized to discard message, to generate arbitrary messages, to concatenate messages together, and to apply cryptographic operation using available keys. The model is described as follows. 

\begin{Definition} A penetrator trace is \emph{one of the following}:
\begin{itemize}
\item[\textbf{M}.] Text message: $\langle+t\rangle$ where $t \in T$
\item[\textbf{F.}] Flushing: $\langle-g\rangle$ 
\item[\textbf{T.}] Tee: $\langle-g,+g,+g\rangle$
\item[\textbf{C.}] Concatenation $\langle-g,-h,+gh\rangle$
\item[\textbf{S.}] Separation into components: $\langle-gh,+g,+h\rangle$
\item[\textbf{K.}] Key: $\langle+K\rangle$ where $K \in \mathcal{K_P}$
\item[\textbf{E.}] Encryption: $\langle-K,-h,+\{h\}_K\rangle$
\item[\textbf{D.}] Decryption: $\langle-K^{-1},-\{h\}_K,+h\rangle$
\end{itemize} 
\end{Definition}

This penetrator's trace set given here could be extended if desired without any modification on the whole model. However, the proofs should be adjusted to take into account the additional penetrator traces. This ability gives us an open space to add some physical penetrator traces without worry of proving way. 

\section{Our Extension of Strand Spaces}

\subsection{Extension to Algebra}

Our definition of Strand Spaces algebra is based on the definition from~\cite{1212716}, which adds to the model a possibility to deal with DH operations, hash functions, and signatures. To take into account device pairing protocols, we do not need to consider signatures (neither asymmetric encryption), but must add keyed hash function, or MAC function. We thus redefine the set of terms as follows:

\begin{Definition}
The set of \emph{terms} $\mathcal{A}$ is assumed to be freely generated from four disjoint sets: predictable texts $\mathcal{T}$, unpredictable texts $\mathcal{R}$, keys $\mathcal{K}$, and Diffie-Hellman values $\mathcal{D}$. 

The set of keys $\mathcal{K}$ is divided into two disjoint sets: verification keys $\mathcal{K}_{Ver}$, and keys for symmetric encryption $\mathcal{K}_{Sym}$.

\end{Definition}

\begin{Definition}
\emph{Compound terms} are built by these operations:
\begin{itemize} 
\item join: $\mathcal{A} \times \mathcal{A} \rightarrow \mathcal{A}$, which represents concatenation of terms. 
\item encr: $\mathcal{K}_{Sym} \times \mathcal{A} \rightarrow \mathcal{A}$, which represents encryption. 
\item DH: $\mathcal{D} \times \mathcal{D} \rightarrow \mathcal{D}$, which represents the Diffie-Hellman operation. We denote the range of DH by $\mathcal{D}_{DH}$.
\item hash: $\mathcal{A} \rightarrow \mathcal{K}_{Sym}$, representing hashing into keys. We denote the range of hash by $\mathcal{K}_{hash}$. 
\item MAC: $\mathcal{K}_{Sym} \times \mathcal{A} \rightarrow \mathcal{K}_{Sym}$, representing MAC operation with a key into keys.
\item commit: $\mathcal{A} \rightarrow (\mathcal{A} \times \mathcal{A})$, producing a data into a commit value and a decommit value.
\end{itemize}
\end{Definition} 

In the following, $encr(k,t)$, $hash(t)$ and $MAC(k,t)$ will be respectively noted ${\{|t|\}}_k$, $h(t)$ and $h_k(t)$. The term $join(t,t')$ will be noted $t,t'$ or $(t,t')$ when necessary to avoid confusion. For a data $t$, commit value is presented at $c(t)$ while decommit value is presented at $d(t)$.

We refine the notions of \textit{subterm} and \textit{component} from previous work on Strand Spaces as follows.

\begin{Definition}[Subterm]
We say that $t$ is a \emph{subterm} of $t'$, written $t \sqsubseteq t'$ if:
\begin{itemize}
\item $t=t'$ or
\item $t'= (t'_1,t'_2)$ then $t \sqsubseteq t'_1$ or $t \sqsubseteq t'_2$,
\item if $t'={\{|t''|\}}_k$, then $t \sqsubseteq t''$,
\item if $t'=h(t'')$, then $t \sqsubseteq t''$,
\item if $t'=h_k(t'')$, then $t \sqsubseteq t''$,
\item if $t'=DH(d_1,d_2)$, then $t \sqsubseteq d_1$ or $t \sqsubseteq d_2$
\item $t' \sqsubseteq c(t)$ for $t \in T$ iff $t' = c(t) $ or $t' \sqsubseteq t$;
\item $t' \sqsubseteq d(t)$ for $t \in T$ iff $t' = d(t)$ or $t' \sqsubseteq t$;
\end{itemize}
\end{Definition}

At last, we introduce the notions of \emph{boxed term} and \emph{revealed term}.

\begin{Definition}[Boxed term] 
For a given bundle, we say that a term $t$ is \emph{boxed} at node $n$, if there exists terms $t'$ and $t''$ such that $t \sqsubseteq t'$, $t' \sqsubseteq term(n)$, and $t'$ has one of the following forms: ${\{|t''|\}}_k$, $h(t'')$, $h_k(t'')$.
\end{Definition}

\begin{Definition}[Revealed term]
For a given bundle, a term $t$ is called to be $revealed$ at node $n$ if:
\begin{itemize}
  \item $t \sqsubseteq term(n)$, and $t$ can be obtained by the penetrator using his knowledge at node $n$, and
  \item for any $n'$ that precedes $n$ ($n' \preceq n$) such that $t \sqsubseteq term(n')$ the penetrator cannot obtain the $t$ using his knowledge at node $n'$. 
\end{itemize} 
\end{Definition}

\subsection{Wireless Strand Spaces}\label{wirelessstrand}

In our extension, we will need to explicitly distinguish between different channels. We thus need to define what is a channel.

\begin{Definition}[Channel] A \emph{channel} is a group of devices which can exchange messages in the same region.
\end{Definition} 

One device may use more than one channel.
For example, given two channels $ch_1$ and $ch_2$ and 3 devices $A$, $B$, $C$, the devices $A$ and $B$ may use $ch_1$, whereas $B$ and $C$ use $ch_2$. 

Playing as a type of secure channels, an out-of-band channel(OOB) is an auxiliary channel between the devices that is both observable and controllable by human that manages the devices. We classify OOB channel in three main types of OOB channels distinguished depending on attacker capacities. Furthermore, the chapter 3 will explain more about OOB channel properties. 

\begin{itemize}
  \item a \textit{private channel} in which attackers cannot overhear, modify, delay or relay messages.
  \item a \textit{protected channel} in which attackers cannot overhear and modify, yet can delay or relay messages. 
  \item a \textit{public channel} in which attackers cannot modify, yet can overhear messages.
\end{itemize}

Additionally we classify public channels into two sub-types: 
\begin{itemize}
  \item a \textit{short-range(SR) public channel} in which attackers cannot delay or relay messages. 
  \item a \textit{long-range(LR) public channel} in which attackers can delay and relay messages. 
\end{itemize}

If no supplementary assumption is declared, a channel is by default an unsecured public wireless channel. Any specific assumption on a channel, must be specified before formalising the protocol. Since a protocol may use several channels, when sending or receiving a term, the used channel must be specified. The definition of signed term is modified in consequence. 

\begin{Definition}[Signed term]
A \emph{signed term} is a triplet $\langle \delta, t, ch \rangle$ , noted $\delta_{ch} t$, where $\delta$ is $+$ (sending) or $-$ (reception), $t$ a term, and $ch$ the channel on which $t$ is sent or received.
\end{Definition}

%Actually, we will see in subsection~\ref{penetrator_model} that the terms manipulated by a penetrator may receive another sign.
By convention, we will specify the channel only when using a specific channel: $-_{ch}t$ means that the term $t$ is received on the channel $ch$, and $-t$ will denote the reception of $t$ on the public wireless channel.

Based on this new definition of signed terms, the definitions of \textit{Strand Spaces}, \textit{node}, \textit{edge}, \textit{originating term}, \textit{uniquely originating term}, and \textit{bundle}, \textit{height of a strand} are the same.

We continue presenting an extension of Strand Spaces model accounting for wireless context, called the \textit{wireless Strand Spaces} accounting for physical aspects of a protocol. Cope with these aspects, an idea is to equip a node in Strand Spaces with \textit{location, timestamp, signal range} values. \footnote{Mobility model can be attached to node definition, however; this model is too complicated to cooperate with verification model. We will consider this problem in close future.} Location shows where the node is standing, timestamp indicates when a node happens, and signal range refers to how far the wireless signal of the node can reach. These values provide a strong muscle for Strand Spaces model to discover several physical attacks such as location bounding, link spoofing... Straightforwardly, the definition of a wireless node is presented as below.

\begin{Definition}[Wireless Node] A \emph{wireless node} $n$ is a tuple of $(t, l_n, t_n, R_n)$ where $t$ is a signed term, $l_n$ is location, $t_n$ is a timestamp, and $R_n$ is signal range.\end{Definition}

Note that, $l_n$, \textit{location of a node n}, is extensible to any Euclidean space. \textit{Distance between nodes} $n$ and $n'$ is noted $dist(l_n,l_n')$ or $dist(n,n')$. $t_n$, or \textit{timestamp of a node n}, appears in a node to check the freshness property, and it is a value of local clock when an event begins. $R_n$, or \textit{signal range of a node n}, is a positive real number. 

In realistic scenario, there always exists a delay between two events, so a fixed value $\delta_{tp}$ is noted as a \textit{processing delay} of edge $ (+n) \Rightarrow (-n')$. We continue describing the definitions of wireless strand, and wireless bundle. 

\begin{Definition}[Wireless Strand] A \emph{wireless strand} is a strand with wireless nodes.
\end{Definition}

Recall that static network is being discussed in this thesis; hence, a strand with fixed location is so-called \textit{a fixed wireless strand}. Thus, all nodes in a fixed strand share the same location, and the same signal range. Given fixed strands $st, st'$ and $ n,n' \in st$, we have $l_n = l_n'$, $R_{n} = R_{n'}$, and $dist(st,st') = dist(n,n')$. A notion $R_{st}$ is indicated as the \textit{signal range of strand st} , since there is no change in radiation power among nodes in the same strand. To avoid ambitious notions, notion of strand in this thesis is now refer to a wireless strand. 

A link is a connection between two participants. In this thesis, four types of links are considered: single logical link, double logical link, single physical link, and double physical link. While a logical link describes existence of path of a term from one strand to another, a physical one describes a directional and physical path without any relaying point from one strand to another one. Moreover, a single link means an unidirectional connection when a double link means a bidirectional one. 

Due to complexity of environments such as in home or in office, existence of a physical link is hard to be correctly criticised. Hence, in this thesis, a physical link is evaluated by a distance value among participants. Precisely speaking, by any mean, a double physical link exists between two participants if and only if the distance between them is lower than their own signal ranges. Formally, we define:

\begin{Definition}
\begin{itemize}
\item \emph{Single physical link}: $\forall st,st' \in \mathcal{B}$, $plink(st,st', \rightarrow) \Leftrightarrow$ $dist(st,st') \le R_{st}$.
\item \emph{Double physical link}: $\forall st,st' \in \mathcal{B}$, $plink(st,st', \leftrightarrow) \Leftrightarrow$ $(dist(st,st') \le R_{st}) \wedge (dist(st',st) \le R_{st'})$.
\item \emph{Single logical link}: $\forall st, st' \in \mathcal{B}, n \in st, n' \in st', \exists (n_1 \rightarrow^* n_2) \Leftrightarrow \exists link(st, st',\rightarrow)$.
\item \emph{Double logical link}: $\forall st, st' \in \mathcal{B},$ $ \exists link(st, st',\rightarrow) \wedge \exists link(st', st,\rightarrow) \Leftrightarrow \exists link(st, st',\leftrightarrow)$.
\end{itemize}	

\end{Definition}

\subsection{Extended Penetrator Model}\label{penndp}

As our extension of fundamental algebra, Strand Spaces model has been broaden with Diffe-Hellman operation, a hash function, and a MAC function. Therefore, concerning on these operations, attackers' capabilities are also extended as well. 

\begin{itemize}
\item \textbf{F}. Fresh DH value: $\langle +d  \rangle$ where $d \in \mathcal{D}_P$ with $\mathcal{D}_P \subset \mathcal{D}$ and $\mathcal{D}_P \cap \mathcal{D}_{DH}=\emptyset$
\item \textbf{H}. Hashing: $\langle -t,+h(t)  \rangle$ 
\item \textbf{MAC}. MAC: $\langle -t,-k, h_k(t)  \rangle$ 
\end{itemize} 

$F$ strand says penetrators can generate their DH values. These values could be used to against the DH-based authentication protocols, in fact. The $H$ strand says penetrators can produce a hash value of any message. And $MAC$ strand says penetrators can produce a MAC value from an arbitrary message and a key. 
 
As described above, OOB channels intentionally limit penetrator's capacities in term of message manipulation. He is prevented from performing actions on private OOB channels, however, he is still able to realise some actions on public or protected OOB channels. We therefore need specific strands to model actions of penetrators on various types of OOB channels. To do so, we extend signed terms with a new event, $\#_ot$, meaning that penetrators suspend the message $t$ on OOB channel $o$. This brand new event is only adopted for public or protected OOB. Consequently, we extend the penetrator model with following two penetrator traces on OOB channels:

\begin{itemize}
\item \textbf{OVH}. Overhearing : $\langle -_om, +m \rangle$ where $o$ is an OOB channel of type public.
\item \textbf{DRP}. Dropping : $\langle -_om \rangle$.
\item \textbf{SUS}. Suspending : $\langle -_om,\#_om \rangle$ where $o$ is a OOB channel of type long-range public or protected. 
\item \textbf{REL}. Releasing : $\langle \#_om,+_om  \rangle$ where $o$ is a OOB channel of type long-range public or protected.
\item \textbf{RPL}. Replaying : $\langle -_om,+_om,+_om  \rangle$ where $o$ is a OOB channel of type long-range public.
\end{itemize} 

$REL$ strand only works for a term $m$ over a long-range public or protected OOB channel $o$ when there exists $SUS$ strand for $m$ over $o$.

Furthermore, since we have tied physical properties in each even of a protocol in previous section, penetrator model would contain some distinct physical attacks. We are interesting in some physical attacks such as link spoofing attack, relaying attack with no delay. 

In \emph{single relaying attack}, an attacker plays as a bridge to connect two honest parties whose distance is longer than their physical signal range. This attack significantly affects to some protocols where distance value is an importance metric for authentication goals. 

\emph{Link spoofing attack} describes that an attacker convinces victims that there are symmetric links among, yet actually the links are just asymmetric ones. To accomplish the attack, the attacker looks for a flawed protocol which does not have any mechanism to verify link status. After finding one, using a high signal power antenna, the attacker create fake links to victims. 

According to above descriptions, the link spoofing attack is constructed from a series of atom actions, called \emph{boosting signal attack}, in which each message is transmitted with high power signal. Consequently, we encode atom malicious actions into new penetrator strands. Given penetrator strand $s_p$, and $n, n' \in s_p$. 
\begin{itemize}
\item[SR.] \emph{Single relay:} \\ $\langle -(t, l_n, t_n, R_n), +(t, l_n', t_{n'}, R_{n'}) \rangle >$ where $t_{n'} - t_n = 0$.
\item [BS.] \emph{Boosting signal:} $\langle +(t, l_n, t_n, R_M) \rangle$ where $R_M$ could be an unlimited value. 
\end{itemize}

At present, single relaying attack is possibly detected by advanced protection mechanisms such as message traveling time estimation, localisation assisted schemes, or directional antenna assisted scheme. Therefore, we consider the \emph{weak penetrator model} be a model which does not deal with relaying attack. In contrast, \emph{strong penetrator model} has full attacker's capabilities. 

\section{Conclusion}

In this chapter, we conceive a formalism as an adaptation of Strand Spaces~\cite{674832}. The model of Strand Spaces is a flexible formalism which represents protocols as a set of local views of participants in a run of a protocol. Our formalism models physical properties in a natural manner, and permits verification of security properties relevant to these protocols. The properties are channels, timestamp, location, and signal range. In next chapters, we will use our model to analyse families of secure device pairing and neighbour discovery protocols.

About comparison to other extensions, it is better to put our model/extensions into particular circumstances where we can directly show the brightness of our model beside others in the same problem. Thus, it is quite unfair if they do not stay on our territories. For example, metric strand~\cite{Thayer:2010aa} only concerns on local authentication problems, yet not on pairing problems. In fact, every single problem in the thesis has its own related work in which we will put our work in a comparison to others. 

Worthily to be noted, without consideration in a protocol, physical information will not be presented at wireless nodes. Particularly in this thesis, the chapter 3 just concerns on channel information, so other information such as location, time, and signal range will not be included. Penetrator strands associated to these information will not be examined as well.




